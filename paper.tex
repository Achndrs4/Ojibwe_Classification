\documentclass{report}

%% Language and font encodings
\usepackage[english]{babel}
\usepackage[utf8x]{inputenc}
\usepackage[T1]{fontenc}
% * <achndrs4@illinois.edu> 2017-03-07T16:20:59.141Z:
%
% ^.

%% Sets page size and margins
\usepackage[top=3cm,bottom=2cm,left=3cm,right=3cm,marginparwidth=1.75cm]{geometry}

%% Useful packages
\usepackage{amsmath}
\usepackage{graphicx}
\usepackage[colorinlistoftodos]{todonotes}
\usepackage[colorlinks=true, allcolors=black]{hyperref}

% Specify bibliography package
\usepackage{natbib}



\title{Ojibwe(ISO 639-3: oji)}
\author{Anirudh Chandrashekhar}
\date{October 4th 2017}

\begin{document}
\maketitle
The Ojibwe language is an Algonquin language consisting of a dialect continuum\footnote{Some scholars classify Ojibwe into completely different languages,which is why the ISO 639-3 for Ojibwe also includes ciw (Chippewa), ojs (Severn Ojibwe), alq (Algonquin), otw (Ottowa), ojb (Northwestern Ojibwe) and others} that spans northern Wisconsin and Michigan up through Ontario and Southern Saskatchewan in Canada \citep{bloomfield2016eastern}. It has around 8,000 native speakers in the United States, and over 45,000 speakers in Canada;given that the Ojibwe language has no centralized language authority (like Navajo or Nunavut Inuit), there is no agreed orthography across dialects, with most in the US and Southern Canada opting to use a Roman transliteration, and the northern dialects using the same Canadian Aboriginal syllabics that are used in Inuktitut and Cree \citep{bloomfield2016eastern}.

Ojibwe is a photosynthetic language (although less so than, say, Eskimo-Aluet languages) , and is highly verb-inflectional. Ojibwe is a head-marking language syntactically , and is usually VOS, although occasionally VSO, and while word order is usually preferred in these two forms (especially in Eastern Ojibwe) all word orders are attested \citep{bloomfield2016eastern}.Ojibwe words can be classified into four main categories: nouns, pronouns, verbs, and particles. While the nouns, pronouns and verbs can be highly inflected, particles cannot \citep{wiscgrammar}.

Nouns in Ojibwe are split into four grammatical categories- gender, person, obviation, and inclusion \citep{todd1970grammar}. Nouns are highly inflectional, and can possess references to other nouns via inflection. For example,nouns that do not reference a possessor may be deleted when other nouns can convey this information: \emph{ihkwee okosisan} (Gloss: the-woman  her-son) can be reduced to \emph{okosisan}(Gloss:her-son).Nouns are classified as animate or inanimate (which corresponds to gender) based on the gender of their stem word, and every noun, including pronoun, corresponds to one of these genders. As with all grammatical gender, the rules regarding what qualifies as animate and inanimate are mostly arbitrary with pronouns, animals, and some plants being animate, as well as some grammatically animate inanimate objects, like \emph{Aashikan} (foot-wrapping, sock) and \emph{otaahsan}(pants), along with all other articles of clothing \citep{bloomfield2016eastern}. Ojibwean nouns have singular and plural forms, can be differentiated by person, as well as inclusion,- for example nouns like \emph{Kiinawint} (Gloss: We including the speaker) and \emph{kiinawaa} (Gloss: You excluding the first person).

A unique feature in Ojibwe, and a few other Algonquin languages is the use of Obviation in classifying nouns. Obviation differentiates between third persons in a given context. A third person become obviate when there is a grammatical link to a more prominent third person(called a proximate)that has already been mentioned in the sentence. To understand the difference between the two, we can look at the following sentence: \emph{ninkosihsan owaapaman}(Gloss: my-son[proximate] he-sees-him[obv]).
Here, the third speaker (he) is the obviate, connected to the more prominent/significant/closer proximate (my-son) \citep{todd1970grammar}.

Ojibwe also has simple particles, \emph{kaa}(who will) and \emph{kee}(that will) that are compounded with conjunct verbs and create relative clauses.\citep{bloomfield2016eastern}. 

Verbs in Ojibwe are heavily inflectional as well, even more so than substantives. Verbs can also carry this information, causing pronoun-drops in most informal usage of the language. For example, the phrase \emph{Niin niinkihkeentaan} (Gloss: I I-know-it) can be reduced to  \emph{niinkihkeentaan} (Gloss: I know it). A verb consists of a verb stem with inflectional affixes, and the order in which these affixes are added on indicate which mood the verb takes: the order of \emph{Kinipaa} (Ke: second person + Nipaa: sleep) indicates that the verb is in the declarative mood (you are sleeping), while \emph{nipaan} (Nipaa:sleep + n: second person) is an imperative(sleep!). Voice defines the subject-object relationship between two individuals in a verb, and is naturally expressed in only transitive verbs. There are three modes in  Ojibwe, preterit, dubitative, and negative (with present being unmarked), and can be chained together:compare \emph{anohkii-pan} (Dubious: He is probably working) with \emph{ahnokiikopan} (Dubious-Preterit: He was probably working) and even \emph{kaawin anohkiihsiikopan} (Dubious-Preterit-Negative: He probably wasn't working)\citep{todd1970grammar}.



 

\bibliographystyle{apalike}
\bibliography{references}

\end{document}